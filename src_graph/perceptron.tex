\documentclass{article}

% Language setting
% Replace `english' with e.g. `spanish' to change the document language
\usepackage[utf8x]{inputenc} % Включаем поддержку UTF8
\usepackage[russian]{babel}

% Set page size and margins
% Replace `letterpaper' with `a4paper' for UK/EU standard size
\usepackage[a4paper,top=2cm,bottom=2cm,left=3cm,right=3cm,marginparwidth=1.75cm]{geometry}

% Useful packages
\usepackage{amsmath}
\usepackage{graphicx}
\usepackage[colorlinks=true, allcolors=blue]{hyperref}

\title{MLP perceptron}
\author{nohoteth, almetate}

\begin{document}
\maketitle

\section{Введение}
MLP perceptron :
    \begin{itemize}
        \item Предназначена для формирования и обучения модели нейронных сетей для классификации рукописных букв латинского алфавита
        \item Интерфейс программы предоставляет возможность:
        \begin{enumerate}
            \item классифицирование изображения с рукописными буквами латинского алфавита;
            \item запустить эксперимент на тестовой выборке или ее части, задаваемой дробным числом от 0 до 1 (где 0 - это пустая выборка - вырожденная ситуация, а 1 - вся тестовая выборка целиком). После выполнения эксперимента на экран должно быть выведена средняя точность (average accuracy), прецизионность (precision), полнота (recall), f-мера (f-measure) и общее затраченное время
            \item загружать BMP-изображения (размер изображения может достигать 512x512) с буквами латинского алфавита и осуществлять их классификацию
            \item рисовать двухцветные квадратные изображения от руки в отдельном окне
            \item запускать процесс обучения в реальном времени для заданного пользователем количества эпох с выводом на экран контрольных значений ошибки для каждой эпохи обучения. Предусмотреть возможность составление отчета в виде графика изменения ошибки, посчитанной на тестовой выборке, для каждой эпохи обучения
            \item запускать процесс обучения с применением кросс-валидации для заданного числа групп k
            \item переключать реализацию перцпетрона (матричная или графовая)
            \item переключать количество скрытых слоев перцептрона (от 2 до 5)
            \item сохранять в файл и загружать из файла веса перцептрона
            \item использует сигмоидальную функцию активации для каждого скрытого слоя;
            \item имеет возможность обучаться на открытом датасете (например, EMNIST-letters представленным в директории datasets);
            \item обучается с использованием метода обратного распространения ошибки;
        \end{enumerate}
    \end{itemize}

\section{Установка}

\begin{enumerate}
    \item Скачайте репозиторий проекта;
    \item Перейдите в терминале в папку src проекта;
    \item Выполните \textbf{make install};
    \item Откройте папку приложения на рабочем столе или в домашней дирректории и запустите приложение.
\end{enumerate}

\section{Удаление}

\begin{enumerate}
    \item Перейдите в терминале в папку src проекта;
    \item Выполните: \textbf{make uninstall};
    \item Удалите папку проекта.
\end{enumerate}

\end{document}
